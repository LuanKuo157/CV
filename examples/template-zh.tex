%% start of file `template-zh.tex'.
%% Copyright 2006-2013 Xavier Danaux (xdanaux@gmail.com).
%
% This work may be distributed and/or modified under the
% conditions of the LaTeX Project Public License version 1.3c,
% available at http://www.latex-project.org/lppl/.


\documentclass[10pt,a4paper,sans]{moderncv}   % possible options include font size ('10pt', '11pt' and '12pt'), paper size ('a4paper', 'letterpaper', 'a5paper', 'legalpaper', 'executivepaper' and 'landscape') and font family ('sans' and 'roman')

% moderncv 主题
\moderncvstyle{classic}                        % 选项参数是 ‘casual’, ‘classic’, ‘oldstyle’ 和 ’banking’
\moderncvcolor{orange}                          % 选项参数是 ‘blue’ (默认)、‘orange’、‘green’、‘red’、‘purple’ 和 ‘grey’
%\nopagenumbers{}                             % 消除注释以取消自动页码生成功能

% 字符编码
\usepackage[utf8]{inputenc}                   % 替换你正在使用的编码
\usepackage{amssymb}
\usepackage{CJKutf8}

% 调整页面
%\usepackage[scale=0.75]{geometry}
\usepackage[scale=0.9]{geometry}
\setlength{\hintscolumnwidth}{2.8cm}           % 如果你希望改变日期栏的宽度

% 个人信息
\name{周}{立}
\title{个人简历/RESUME}                     % 可选项、如不需要可删除本行
\phone[mobile]{+86~136~3707~1255}              % 可选项、如不需要可删除本行
\address{安徽省合肥市金寨路96号 中国科大 230026}            % 可选项、如不需要可删除本行
%\phone[fixed]{+2~(345)~678~901}               % 可选项、如不需要可删除本行
%\phone[fax]{+3~(456)~789~012}                 % 可选项、如不需要可删除本行
\email{zhouli35@mail.ustc.edu.cn}                    % 可选项、如不需要可删除本行
\homepage{home.ustc.edu.cn/$\sim$zhouli35/homepage/}                  % 可选项、如不需要可删除本行
\social[github]{ LuanKuo157}                 % 可选项、如不需要可删除本行
\extrainfo{性别 男 \\
民族 汉 \\
出生日期 1990/10/28 \\      
籍贯 湖北黄冈 \\
政治面貌 党员}                 % 可选项、如不需要可删除本行
\photo[64pt][0.6pt]{picture}                  % ‘64pt’是图片必须压缩至的高度、‘0.4pt‘是图片边框的宽度 (如不需要可调节至0pt)、’picture‘ 是图片文件的名字;可选项、如不需要可删除本行
\quote{求职意向/Objective\\
Seeking a position in solid geophysics or geophysical prospecting that offer stability and growth opportunities in a dynamic organization.}                          % 可选项、如不需要可删除本行

% 显示索引号;仅用于在简历中使用了引言
%\makeatletter
%\renewcommand*{\bibliographyitemlabel}{\@biblabel{\arabic{enumiv}}}
%\makeatother

% 分类索引
%\usepackage{multibib}
%\newcites{book,misc}{{Books},{Others}}
%----------------------------------------------------------------------------------
%            内容
%----------------------------------------------------------------------------------
\begin{document}
\begin{CJK}{UTF8}{gbsn}                       % 详情参阅CJK文件包
\maketitle

\section{教育背景}
\cventry{2009.9 -- 2013.7 }{地球物理学学士}{长安大学地质工程和测绘学院}{西安}{\textit{成绩专业前10\%}}{毕业论文 \textbf{地震早期预警中的实时动态定位研究} 导师:白超英 教授}
\cventry{2013.9 -- 至今}{地球物理学硕士}{中国科学技术大学地球和空间科学学院}{合肥}{}{研究方向 \textbf{波动方程层析成像 全波形反演 数据处理 震源机制反演} 导师:张伟 教授} 

\section{主要实习及工作经历}
\cvlistitem{2013.7 -- 2013.8 武汉金钥匙教育 高中暑期培训班 班主任}
\cvlistitem{2014.9 -- 2015.1 中国科学技术大学 勘探地震学进展 助教}
%\section{毕业论文}
%\cvitem{题目}{\emph{题目}}
%\cvitem{导师}{导师}
%\cvitem{说明}{\small 论文简介}
%
%\section{工作背景}
%\subsection{专业}
%\cventry{年 -- 年}{职位}{公司}{城市}{}{不超过1--2行的概况说明\newline{}%
%工作内容:%
%\begin{itemize}%
%\item 工作内容 1;
%\item 工作内容 2、 含二级内容:
%  \begin{itemize}%
%  \item 二级内容 (a);
%  \item 二级内容 (b)、含三级内容 (不建议使用);
%    \begin{itemize}
%    \item 三级内容 i;
%    \item 三级内容 ii;
%    \item 三级内容 iii;
%    \end{itemize}
%  \item 二级内容 (c);
%  \end{itemize}
%\item 工作内容 3。
%\end{itemize}}
%\cventry{年 -- 年}{职位}{公司}{城市}{}{说明行1\newline{}说明行2}
%\subsection{其他}
%\cventry{年 -- 年}{职位}{公司}{城市}{}{说明}

\section{个人技能}
\subsection{语言技能}
\cvlistdoubleitem{英语 CET-6 2010.12 分数497} 
{可以基本无障碍的口语交流,有英文写作、报告和答辩的经历,并有一次英文专业会议的同声传译经历}
%cvitemwithcomment{语言 2}{水平}{评价}
%cvitemwithcomment{语言 3}{水平}{评价}

\subsection{计算机技能}
\cvlistdoubleitem{2010年通过计算机二级(C)}{掌握Fortran,C,Matlab,HTML,Shell,Python脚本等编程语言,略通MPI并行计算}
\cvlistdoubleitem{熟悉linux操作和服务器运维}{熟悉office办公软件,掌握latex文档编辑的基本操作}
\cvlistdoubleitem{地震勘探数据处理软件Geotomo, Vista基本操作}{熟悉SAC,GMT,Surfer,Grapher等软件的使用}
\section{荣誉奖励}
\cvitemwithcomment{2010年 10月}{国家励志奖学金}{奖项级别:国家级}
\cvitemwithcomment{2011年 10月}{国家励志奖学金}{奖项级别:国家级}
\cvitemwithcomment{2012年 10月}{国家励志奖学金}{奖项级别:国家级}
\cvitemwithcomment{2013年 04月}{全国大学生数学竞赛国家三等奖}{奖项级别:国家级}
\cvitemwithcomment{2013年 06月}{长安大学优秀毕业生}{奖项级别:院校级}
\cvitemwithcomment{2015年 06月}{东方杯全国大学生勘探地球物理大赛国家三等奖}{奖项级别:国家级}
~~~~~~~\small{其他体育、书法类等小奖项不一一列举。}
\section{兴趣爱好}
\renewcommand{\listitemsymbol}{*}             % 改变列表符号
\cvlistdoubleitem{跑步}{ 参加横店、合肥马拉松}
\cvlistdoubleitem{篮球,硬笔书法}{喜欢关注Github 开源软件项目}
\section{自我评价}
本人性格稳重、温和有活力,待人真诚,有一定的团队组织和协作能力。研究生期间除了自己的科研工作外还负责了组内大多数出差、外业以及
参会等事项,纪律性强,工作认真负责。有较强的抗压能力,研究生期间的研究方向是一个难度较大且耗时长的课题,但始终坚持要有始有终,做出结果。


%\renewcommand{\listitemsymbol}{-}             % 改变列表符号

%\section{其他 2}
%\cvlistdoubleitem{项目 1}{项目 4}
%\cvlistdoubleitem{项目 2}{项目 5\cite{book1}}
%\cvlistdoubleitem{项目 3}{}

% 来自BibTeX文件但不使用multibib包的出版物
%\renewcommand*{\bibliographyitemlabel}{\@biblabel{\arabic{enumiv}}}% BibTeX的数字标签
%\nocite{*}
%\bibliographystyle{plain}
%\bibliography{publications}                    % 'publications' 是BibTeX文件的文件名

% 来自BibTeX文件并使用multibib包的出版物
%\section{出版物}
%\nocitebook{book1,book2}
%\bibliographystylebook{plain}
%\bibliographybook{publications}               % 'publications' 是BibTeX文件的文件名
%\nocitemisc{misc1,misc2,misc3}
%\bibliographystylemisc{plain}
%\bibliographymisc{publications}               % 'publications' 是BibTeX文件的文件名

\clearpage\end{CJK}
\end{document}


%% 文件结尾 `template-zh.tex'.
